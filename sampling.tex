%!TEX root = main.tex

% Sampling Section of negFE

\section{Sampling Proceedure}
The sampling proceedure that we will use and discuss in this work is two fold.
There is first, a family of distributions that we will call $\mathcal{W}$. 
There is some finite number of distributions in this family, we call $|\mathcal{W}| = \mathcal{R}$. 
We let $Z$ be an index for the distributions in the family and we denote the $Z$th distribution $W_Z$. 
The distribution $W_Z$ can then be sampled, and we denote a sample of $W_Z$ as $w_{Z} \in \zeroone{n}$. When $Z$ is clear, we will omit the subscript.

To sample a uniform point in $\mathcal{W}$, we can uniformly select $Z$ and then pick from $W_Z$ unifromly. 
This two stage process gives us more tools to reason about the ability of an efficient adversary. 

In general, we will assume that $\mathcal{W}$ is public to any adversary, algorithm, or party we may discuss. 
We will generally assume that the specific selection of $Z$ is only shared with specific parties.
We will also assume that the specific $w$ sampled from $W_Z$ is private and only known to parties who are given it explicity. 

An interesting part of this work is how we share information about $W_Z$. We will allow an inefficient adversary access to the entire description of $W_Z$ and ask them to produce an advice string for an efficient adversary, that is that the advice string length is bound from above by a arbitrary polynomial function of our security parameter. 

\subsection{Distributions over $\zeroone{n}$}
\begin{description}
    \item[$U_{n,k}$] Picking $k$ points from $\zeroone{n}$ without replacement uniformly results in a distribution we will denote $U_{n,k}$. Note that there are ${2^{n}\choose k}$ outcomes each occuring with probability $1/{2^{n}\choose k}$.

%    \item[$U_{n,k,t}^{-}$] Another distribution of interest is picking $k$ points uniformly from $\zeroone{n}$ and then removing points that are within distance $t$ of one another, we denote this $U_{n,k,t}^{-}$. 
    \item[$\mathtt{Code}_{n,k,t}^{*}$] The set of all $k$ points where all points are distance $t$ from one another.  That is, the set of all $(n,\log k, t)$ error-correcting codes. The trivial bounds from above apply here as well.  Note for all outcomes $X$ from this distribution $\Hfuzz(X) = \log k$.
    
    \item[$\mathtt{PCode}_{n, k, t, \alpha}^{*}$] The uniform distribution over all sets $X = \{x_i\}$ of size $k$ with the additional condition that $\Hfuzz(X) \ge \alpha$.  Note that this weakens the condition of $\mathtt{Code}$ above as some distributions are allowed to lie within distance $t$ as long as fewer than $2^{-\alpha}*k$ do lie within a single ball of radius $t$.
\end{description}
 
 
 \begin{lemma}
 \label{lem:close family}
 $\Delta(U_{n,k}, \mathtt{PCode}_{n, k, t, \alpha}^{*}) \le 2^{n+k\log e- nk2^{-\alpha}(1-h_2(t/n))}$.
 \end{lemma}
 
 \begin{proof} We begin by recalling the bounds that $(a/b)^b \le {a\choose b} \le (ae/b)^b$.
 We begin by noting that the size of the set of ${2^n \choose k}$ points that $U_{n,k}$ is uniformly sampled from
 \begin{align*}
 {2^n\choose k} \ge 2^{nk}/k^k.
 \end{align*}  We analyze what fraction of members of $U_{n,k}$ are not in $\mathtt{PCode}_{n, k, t, \alpha}^{*}$.
 We begin by enumerating the number of entries that are not in $\mathtt{PCode}$:
 \[
  \underbrace{{2^n \choose 1}}_{(1)}  \underbrace{{|B_t| \choose 2^{-\alpha}k}}_{(2)}   \underbrace{{2^n \choose k(1-2^{-\alpha})}}_{(3)}
 \]
 In the above, 
 \begin{enumerate}
 \itemsep0em
 \item The choice of the point for $w^*$ that will be the center of the ball.
 \item The choice for the at least $2^{-\alpha} k$ points that need to be within distance $t$ of the center point.
 \item The choice for the remaining points $k(1-2^{-\alpha})$ that need to be somewhere in the space outside of the ball.  
 \end{enumerate}
 Now one has: 
 \begin{align*}
|U_{n, k} \setminus \mathtt{PCode}_{n, k, t, \alpha}^{*}|  &=2^n \cdot {|B_t| \choose 2^{-\alpha}k}\cdot {2^n \choose k(1-2^{-\alpha})}\\
&= 2^n \cdot {2^{nh_2(t/n)} \choose 2^{-\alpha}k}\cdot {2^n \choose k(1-2^{-\alpha})}\\
&\le 2^n \cdot \left(\frac{e2^{nh_2(t/n)}}{2^{-\alpha}k}\right)^{2^{-\alpha}k}\cdot \left(\frac{e2^n}{k(1-2^{-\alpha})}\right)^{k(1-2^{-\alpha})}\\
&\le 2^{n+k\log e} \frac{2^{nk(1-2^{-\alpha}(1-h_2(t/n)))}}{(2^{-\alpha}k)^{2^{-\alpha}k} (k(1-2^{-\alpha}))^{k(1-2^{-\alpha}})}\\
 \end{align*}
 \todo{This needs to be tighter}
 Thus, one has that 
 \begin{align*}
\Delta(U_{n,k}, \mathtt{PCode}_{n, k, t, \alpha}^{*}) &= \frac{|U_{n, k} \setminus \mathtt{PCode}_{n, k, t, \alpha}^{*}|}{|U_{n, k}|}\\
&\le\frac{2^{n+k\log e} \frac{2^{nk(1-2^{-\alpha}(1-h_2(t/n)))}}{(2^{-\alpha}k)^{2^{-\alpha}k} (k(1-2^{-\alpha}))^{k(1-2^{-\alpha}})}}{\frac{2^{nk}}{k^k}}\\
&\le\frac{2^{n+k\log e} \frac{2^{nk} 2^{-nk2^{-\alpha}(1-h_2(t/n)))}}{2^{-\alpha 2^{-\alpha k}}k^kk^{-k2^{-\alpha}}}}{\frac{2^{nk}}{k^k}}\\
&\le \frac{2^{n+k\log e}}{2^{nk2^{-\alpha}(1-h_2(t/n))}} = 2^{n+k\log e- nk2^{-\alpha}(1-h_2(t/n))}.
 \end{align*}
  \end{proof}