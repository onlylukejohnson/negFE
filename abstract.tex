%!TEX root = main.tex

\abstract{
Fuzzy extractors convert noisy signals from the physical world into reliable cryptographic keys.  An important open question is building fuzzy extractors that secure all possible distributions.  The relevant notion of distribution quality is fuzzy min-entropy which measures the probability of nearby points which will be transformed to the correct key (Fuller, Reyzin, and Smith, IEEE Transactions on Information Theory 2020).
 
There is a wide gap between what is known with security against computational and information-theoretic adversaries.  Under strong computational assumptions, there is a single fuzzy extractor that works for all distributions with fuzzy min-entropy.  However, with information-theoretic security, it is impossible to build a single fuzzy extractor that works for all distributions (Fuller, Reyzin, and Smith, IEEE Transactions on Information Theory 2020). A weaker goal is to build a fuzzy extractor with information-theoretic security for each probability distribution.  This is the approach taken by Woodage et al. (Crypto 2017).  Approaches that follow this strategy require the full description of the probability mass function and are inefficient.

We show this is inherent: for most distributions with $2^k$ points there is no fuzzy extractor that is secure whose information about the probability mass function is less than $\Theta(k2^k)$.  This result rules out the possibility of efficient, information-theoretic fuzzy extractors for most distributions with fuzzy min-entropy.

We show an analogous result with strong parameters for information-theoretic secure sketches which are frequently used to construct fuzzy extractors. 
}