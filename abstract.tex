%!TEX root = main.tex

\abstract{
Fuzzy extractors convert noisy signals from the physical world into reliable cryptographic keys.  %An important open question is building fuzzy extractors that secure all possible distributions.  
Fuzzy min-entropy upper bounds the length of the key that can be derived from a distribution (Fuller, Reyzin, and Smith, IEEE Transactions on Information Theory 2020).  Specifically, fuzzy min-entropy is proportional to the sum of the probability of nearby points, such points are transformed to the correct key and suffice for adversary success. Fuzzy min-entropy that is superlogarithmic in the security parameter is required for a noisy distribution to be suitable for key derivation.
 
There is a wide gap between what is possible with respect to computational and information-theoretic adversaries.  A good obfuscator derives keys from all distributions with superlogarithmic entropy.

Against information-theoretic adversaries, it is impossible to build a single fuzzy extractor that works for all distributions (Fuller, Reyzin, and Smith, IEEE Transactions on Information Theory 2020). A weaker goal is to build a fuzzy extractor for each probability distribution.  This is the approach taken by Woodage et al. (Crypto 2017).  Prior approaches use the full description of the probability mass function and are inefficient.
We show this is inherent: \textbf{for most distributions with $2^k$ points there is no secure fuzzy extractor that uses less  $\Theta(k2^k)$ bits of information about the distribution.}  This result rules out the possibility of efficient, information-theoretic fuzzy extractors for most distributions with fuzzy min-entropy.

We show an analogous result with stronger parameters for information-theoretic secure sketches. Secure sketches are frequently used to construct fuzzy extractors. 
}