\def\lncs{0}


\ifnum\lncs=0
  \documentclass[11pt]{article}
  \usepackage[top=3cm, bottom=3cm, left=2cm, right=2cm]{geometry}      % [top=2cm, bottom=2cm, left=2cm, right=2cm]
  \geometry{letterpaper}                   % ... or a4paper or a5paper or ...
  %\geometry{landscape}                % Activate for for rotated page geometry
  %\usepackage[parfill]{parskip}
  \usepackage{amsthm}
  \newtheorem{theorem}{Theorem}
  \newtheorem{lemma}[theorem]{Lemma}
  \newtheorem{proposition}[theorem]{Proposition}
  \newtheorem{corollary}[theorem]{Corollary}
  \newtheorem{conjecture}{Conjecture}
  \newtheorem{definition}{Definition}
  \newtheorem{assumption}{Assumption}
  \newtheorem{claim}{Claim}
  \newtheorem{problem}{Problem}
  \newtheorem{construction}{Construction}
  \newtheorem{remark}{Remark}
  \renewcommand{\vec}[1]{\vect{#1}}      % activate for lncs style vectors

\else
  \documentclass[a4paper,UKenglish,cleveref, autoref, thm-restate]{lipics-v2021}
  \newtheorem{construction}{Construction}
  \renewcommand{\paragraph}[1]{\subsubsection{#1}}
\fi

\usepackage[dvipsnames]{xcolor}

\usepackage{tikz}
\usepackage{graphicx}
\usepackage{csquotes}
\usepackage{amssymb, amsmath, amsfonts}
\usepackage{enumerate}
\usepackage{hyperref}
\usepackage{xspace}
\usepackage{graphicx}
\usepackage{latexsym}
\usepackage{color}
\usepackage{framed}
\usepackage{algpseudocode}
\usepackage{breakcites}

\mathchardef\mhyphen="2D
\newcommand{\secref}[1]{\mbox{Section~\ref{#1}}}
\newcommand{\subsecref}[1]{\mbox{Subsection~\ref{#1}}}
\newcommand{\apref}[1]{\mbox{Appendix~\ref{#1}}}
\newcommand{\thref}[1]{\mbox{Theorem~\ref{#1}}}
\newcommand{\exref}[1]{\mbox{Example~\ref{#1}}}
\newcommand{\defref}[1]{\mbox{Definition~\ref{#1}}}
\newcommand{\corref}[1]{\mbox{Corollary~\ref{#1}}}
\newcommand{\lemref}[1]{\mbox{Lemma~\ref{#1}}}
\newcommand{\assref}[1]{\mbox{Assumption~\ref{#1}}}
\newcommand{\probref}[1]{\mbox{Problem~\ref{#1}}}
\newcommand{\clref}[1]{\mbox{Claim~\ref{#1}}}
\newcommand{\propref}[1]{\mbox{Proposition~\ref{#1}}}
\newcommand{\remref}[1]{\mbox{Remark~\ref{#1}}}
\newcommand{\consref}[1]{\mbox{Construction~\ref{#1}}}
\newcommand{\figref}[1]{\mbox{Figure~\ref{#1}}}
\newcommand{\conditionalpara}[1]{
\ifnum\lncs=1\noindent \textbf{#1} \else \paragraph{#1}\fi}
\newcommand{\conditionaleqn}[1]{\ifnum\lncs=1$#1$\else \[#1\]\fi}
\DeclareMathOperator*{\expe}{\mathbb{E}}
\DeclareMathOperator*{\var}{\text{Var}}
\DeclareMathOperator*{\Exp}{\mathbb{E}}
\DeclareMathOperator*{\argmax}{arg\,max}
\DeclareMathOperator*{\argmin}{arg\,min}

% Commands
\newcommand{\comment}[1]{\textcolor{Mulberry}{#1}}
\newcommand{\todo}[1]{\textcolor{red}{TODO: #1}}
%\newcommand{\todo}[1]{}
\newcommand{\here}{\todo{Stopped here! }}

%Probability
\newcommand{\Ex}[1]{\Exp\left[#1\right]}
\newcommand{\Exlim}[2]{\Exp\limits_{#1}\br{#2}}
\newcommand{\Prob}[1]{\Pr\br{#1}}
\newcommand{\Problim}[2]{\Pr\limits_{#1}\br{#2}}
\newcommand{\p}[1]{p\prns{#1}}

% Entropy
\newcommand{\ent}[1]{\mathrm{H}\prns{#1}}
\newcommand{\minent}[1]{\mathrm{H}_{\infty}\prns{#1}}
\newcommand{\acminent}[2]{\tilde{\mathrm{H}}_{\infty}\prns{#1|#2}}
\newcommand{\hart}[1]{\mathrm{H}_0\prns{#1}}


%MACROS galore!!!
\newcommand{\class}[1]{{\ensuremath{\mathsf{#1}}}}
\newcommand{\gen}{\ensuremath{\class{Gen}}\xspace}
\newcommand{\aux}{\ensuremath{\class{Advise}}\xspace}
\newcommand{\advise}{\ensuremath{\class{advise}}\xspace}
\newcommand{\rep}{\ensuremath{\class{Rep}}\xspace}
\newcommand{\sketch}{\ensuremath{\class{SS}}\xspace}
\newcommand{\viable}{\ensuremath{\mathtt{Viable}}\xspace}
\newcommand{\rec}{\ensuremath{\class{Rec}}\xspace}
\newcommand{\enc}{\ensuremath{\class{Enc}}\xspace}
\newcommand{\wt}{\ensuremath{\textsf{wt}}\xspace}
\newcommand{\dec}{\ensuremath{\class{Dec}}\xspace}
\newcommand{\prg}{\ensuremath{\class{prg}}\xspace}
\newcommand{\zo}{\ensuremath{\{0, 1\}}}
\newcommand{\vect}[1]{\ensuremath{\mathbf{#1}}}
\newcommand{\zq}{\ensuremath{\mathbb{Z}_q}}
\newcommand{\Fq}{\ensuremath{\mathbb{F}_q}}
\newcommand{\sample}{\ensuremath{\class{Sample}}\xspace}
\newcommand{\neigh}{\ensuremath{\class{Neigh}}\xspace}
\newcommand{\error}{\ensuremath{\class{Err}}\xspace}
\newcommand{\weight}{\ensuremath{\class{Wgt}}\xspace}
\newcommand{\dis}{\ensuremath{\mathsf{dis}}}
\newcommand{\bin}{\ensuremath{\mathsf{Bin}}}
\newcommand{\decode}{\ensuremath{\mathsf{Decode}}}
\newcommand{\guess}{\mathsf{guess}}
\newcommand{\nullsp}{\mathtt{null}}


\newcommand{\A}{\mathcal{A}}
\newcommand{\F}{\mathbb{F}}


\newcommand{\metric}{\ensuremath{\mathtt{Metric}}\xspace}
\newcommand{\bdde}{\ensuremath{\mathtt{BDDE}}\xspace}
\newcommand{\bdders}{\ensuremath{\mathtt{BDDE-RS}}\xspace}
\newcommand{\bdderl}{\ensuremath{\mathtt{BDDE-RL}}\xspace}
\newcommand{\findrep}{\ensuremath{\mathtt{FIND-REP}}\xspace}
\newcommand{\hill}{\ensuremath{\mathtt{HILL}}\xspace}
\newcommand{\hillrlx}{\ensuremath{\mathtt{HILL\mhyphen rlx}}\xspace}
\newcommand{\yao}{\ensuremath{\mathtt{Yao}}\xspace}
\newcommand{\unp}{\ensuremath{\mathtt{unp}}\xspace}
\newcommand{\unprlx}{\ensuremath{\mathtt{unp\mhyphen rlx}}\xspace}
\newcommand{\metricstar}{\ensuremath{\mathtt{Metric}^*}\xspace}
\newcommand{\metricd}{\ensuremath{\mathtt{Metric}^*\mathtt{-d}}\xspace}
\newcommand{\hillstar}{\ensuremath{\mathtt{HILL}^*}\xspace}
\newcommand{\hillprime}{\ensuremath{\mathtt{HILL'}}\xspace}
\newcommand{\metricprime}{\ensuremath{\mathtt{Metric'}}\xspace}
\newcommand{\metricprimestar}{\ensuremath{\mathtt{Metric'}^*}\xspace}
\newcommand{\hillprimestar}{\ensuremath{\mathtt{HILL'}^*}\xspace}
\newcommand{\poly}{\ensuremath{\mathtt{poly}}\xspace}
\newcommand{\rank}{\ensuremath{\mathtt{rank}}\xspace}
\newcommand{\ngl}{\ensuremath{\mathtt{ngl}}\xspace}
\newcommand{\Hoo}{\mathrm{H}_\infty}
\newcommand{\Hav}{\tilde{\mathrm{H}}_\infty}
\newcommand{\Haveps}{\tilde{\mathrm{H}}_\infty^\epsilon}
\newcommand{\goodsketch}{\ensuremath{\mathsf{GoodSketch}}\xspace}
\newcommand{\goodkey}{\ensuremath{\mathsf{GoodKey}}\xspace}
\newcommand{\Hfuzz}{\mathrm{H}^{\mathtt{fuzz}}_{t,\infty}}
\newcommand{\Wallfuzz}{\mathcal{W}_{\mathtt{fuzz}}^{\mathtt{all}}}
\newcommand{\Huse}{\mathrm{H}_{\mathtt{usable}}}
\newcommand{\Dom}{\mathsl{Dom}}
\newcommand{\Range}{\mathsl{Rng}}
\newcommand{\Keys}{\mathsl{Keys}}
\def\col{\mathrm{Col}}




% Useful Notation
\newcommand{\defined}{:=}
\newcommand{\sbr}[1]{ \!\left\{ #1 \right\} }
\newcommand{\br}[1]{\!\left[#1\right]}
\newcommand{\prns}[1]{\!\left(#1\right)}
\renewcommand{\log}[1]{\mathsf{log}\prns{#1}}
\newcommand{\zeroone}[1]{\sbr{0,1}^{#1}}

\title{Most Distributions have no Efficient Information-Theoretic Fuzzy Extractors}
\author{Luke Demarest\\University of Connecticut\\luke.h.demarest@gmail.com \and Benjamin Fuller\\University of Connecticut\\benjamin.fuller@uconn.edu\and Alexander Russell\\University of Connecticut\\acr@uconn.edu}
\date{\today}

% Document
\begin{document}
  
  \maketitle
  %!TEX root = main.tex

\abstract{
Fuzzy extractors convert noisy signals from the physical world into reliable cryptographic keys.  Fuzzy min-entropy measures the limit of the length of key that a fuzzy
extractor can derive from a distribution (Fuller, Reyzin, and
Smith, IEEE Transactions on Information Theory 2020).
%Specifically, fuzzy
%min-entropy is proportional to the sum of the probability of nearby
%points, such points are transformed to the correct key and suffice for
%adversary success.
In general, fuzzy min-entropy that is superlogarithmic in the security
parameter is required for a noisy distribution to be suitable for key
derivation.
 
There is a wide gap between what is possible with respect to
computational and information-theoretic adversaries.  Under the
assumption of general-purpose obfuscation, keys can be securely
derived from all distributions with superlogarithmic entropy.
%% I am confused about "obfuscator" here.
Against information-theoretic adversaries, however, it is impossible
to build a single fuzzy extractor that works for all distributions
(Fuller, Reyzin, and Smith, IEEE Transactions on Information Theory
2020).

A weaker information-theoretic goal is to build a fuzzy extractor for
each particular probability distribution.  This is the approach taken
by Woodage et al. (Crypto 2017).  Prior approaches use the full
description of the probability mass function and are inefficient.  We
show this is inherent: \textbf{for a quarter of distributions with
  fuzzy min-entropy and $2^k$ points there is no secure fuzzy
  extractor that uses less $2^{\Theta(k)}$ bits of information about
  the distribution.}  This result rules out the possibility of
efficient, information-theoretic fuzzy extractors for many
distributions with fuzzy min-entropy.

We show an analogous result with stronger parameters for information-theoretic secure sketches. Secure sketches are frequently used to construct fuzzy extractors. 
}

%%% Local Variables:
%%% mode: latex
%%% TeX-master: "main"
%%% End:

  %!TEX root = main.tex

% Introduction for negFE
\section{Introduction}
We show a negative result for efficient fuzzy extractors for all distributions with fuzzy min entropy even when given a (quasi) polynomial advice string from an unbounded collaborator. 

We will compare to other works like \cite{fuller2016fuzzy,fuller2019continuous,fuller2020fuzzy}. 
  %!TEX root = main.tex
% Preliminary Definitions and Results for negFE

\section{Preliminaries}
In this section we will introduce existing results to help clarify our place in the literature, provide necessary existing definitions to show what we are borrowing and what we build on, and we will provide new definitions in support of our novel results.  For distributions $X, Y$, $\Delta(X, Y)$ represents the statistical distance between the two distributions.  That is, 
\[
\Delta(X, Y)\overset{def}= \frac{1}{2}\sum_{x \in X} \left| \Pr[X=x] - \Pr[Y=y]\right|.
\]

\subsection{Existing Definitions}

\begin{definition}[Entropy]
    \emph{Shannon Entropy} or simply Entropy, denoted $\ent{X}$, for some discrete random variable $X$ is a measure of how stable the outcomes of the random variable are. It is calculated as \[\ent{X} \defined \sum\limits_{i=1}^n \p{x_i}\log{\p{x_i}}\] where there are $n$ values that $X$ takes and we denote them as $x_i$. 
\end{definition}

\begin{definition}[Min Entropy]
    \emph{Min Entropy}, denoted $\minent{X}$, is a best case measure of the stability of the random variable $X$. It is calculated as \[\minent{X} \defined -\log{\max\limits_{x_i} p(x_i)}.\]  
\end{definition}

\begin{definition}[Average Conditional Min Entropy]
    \emph{Average Conditional Min Entropy}, denoted $\acminent{X}{Y}$ for two random variables $X$ and $Y$ is an average measure of the remaining entropy of the former given the outcome of the latter. It is calculated as \[ \acminent{X}{Y} \defined -\log{\Exlim{y \leftarrow Y}{\max\limits_{x} \Prob{X = x\ |\ Y = y}}}.\] 
\end{definition}

\begin{definition}[Hartley Entropy]
    \emph{Hartley Entropy} also called \emph{Hartley's Function} measures the uncertainty of a random variable in a basic way, measuring the number of outcomes the random variable has with non-zero probability. Formally, $
    \hart{X} = | \sbr{x \in X\,|\, \Prob{X = x} > 0}|.
    $
\end{definition}

\begin{definition}[Markov's Inequality]
    Markov's inequality is a tail bound for random variables that gives an upper bound on the probability of a random variable deviating from its mean. Let $\Pr[X>0] = 1$. Then the following inequality holds for any $\alpha > 0$: 
    \[ 
      \Prob{X \geq \alpha \cdot \Ex{X}} \leq 1/\alpha .
    \]
\end{definition}

\begin{definition}[Secure Sketch~\cite{dodis2008fuzzy}]
For metric space $(\mathcal{M}, \dis)$, a $(\mathcal{M}, \mathcal{W}, \tilde{m}, t)$-\emph{secure sketch} is a pair of algorithms $(\sketch, \rec)$ with the following properties 
\begin{enumerate} 
\itemsep0em
\item \textbf{Correctness} For all $w, w'$ such that $\dis(w, w')$, let $ss \leftarrow \sketch(w)$ then $\rec(w', ss) = w$ with probability $1$. 
\item \textbf{Security} For all distributions $W \in \mathcal{W}$ it holds that $\Hav(W | \sketch(W)) \ge \tilde{m}$.
\end{enumerate}
\end{definition}

\begin{definition}[Fuzzy Extractor~\cite{dodis2008fuzzy}]
For metric space $(\mathcal{M}, \dis)$, a $(\mathcal{M}, \mathcal{W}, \kappa, t, \epsilon)$-\emph{fuzzy extractor} is a pair of algorithms $(\gen, \rep)$ with the following properties 
\begin{enumerate} 
\itemsep0em
\item \textbf{Correctness} For all $w, w'$ such that $\dis(w, w')$, let $r, p \leftarrow \gen(w)$ then $\rep(w', p) = r$ with probability $1$. 
\item \textbf{Security} For all distributions $W \in \mathcal{W}$, let $R, P \leftarrow \gen(W)$ and $U_\kappa$ be a uniformly distributed random variable over $\zo^\kappa$ it holds that $\Delta((R, P), (U_\kappa, P))\le \epsilon.$
\end{enumerate}
\end{definition}

\subsection{Previous Results}
It has been shown that universal fuzzy extractors are impossible in the information theoretic setting. 

\subsection{Average Conditional Min-Entropy Loss}
\begin{lemma}
    \label{lem:conditionalminentloss}
    Let $\vec{X} = (X_1, X_2, \ldots, X_k)$ be independent random variables. 
    Let $Y$ be a random variable arbitrarility correlated with $\vec{X}$. 
    Then 
    \[
        \acminent{\vec{X}}{Y} \geq \sum \minent{X_i} - \hart{Y}
    \]
\end{lemma} 

\begin{proof}
    Since each $X_i$ is independent then $\minent{\vec{X}} = \sum \minent{X_i}$.
    Now, by definition,
    \begin{align*}
        \acminent{\vec{X}}{Y} &= -\log{\Exlim{y \leftarrow Y}{\max\limits_{\vec{x}} \Prob{\Vec{X} = \vec{x}\ |\ Y = y}}} \\
        &= -\log{\sum\limits_{y} \max\limits_{\vec{x}} \Prob{\vec{X} = \vec{x} \ |\ Y = y} \cdot \Prob{Y = y}}\\
        &= -\log{\sum\limits_{y} \max\limits_{\vec{x}} \Prob{\vec{X} = \vec{x} \vee Y = y}}\\
        &\geq -\log{\sum\limits_{y} \max\limits_{\vec{x}, y'} \Prob{\vec{X} = \vec{x}  ^ Y = y'}}\\
        &= -\log{2^{\hart{Y}} \cdot 2^{\minent{\vec{X},Y}}}\\
        &= \minent{\vec{X},Y} - \hart{Y}\\
        &\geq \minent{\vec{X}} - \hart{Y} \\   
        &= \sum \minent{X_i} - \hart{Y}
    \end{align*}
\end{proof}

\subsubsection{Markov Bound for Predictability}
Markov bounds are tail bounds that use Markov's Inequality to bound the probability that a random variable deviates significantly from its expected value. In Markov's inequality, we necessarily lose a multiplicative factor (here called $alpha$) in order to control the probability of the event occuring. When discussing entropy, we are dealing with a log scaled value which makes losing multiplicative factors costly. Instead, we can perform a Markov bound on the predictability scale. In this case, rather than lose a multiplicative factor in entropy, we lose a multiplicative factor in predictability which translates to a small number of bits of entropy lost for the controlled outcomes.

\todo{Just to make sure this is Lemma 2.2a from \cite{dodis2008fuzzy} right? Shouldn't need to reprove.}

\begin{lemma}
    \label{lem:markovpred}
    Let $\vec{X} = (X_1, X_2, \ldots, X_k)$ be independent random variables. Let $Y$ be a random variable arbitrarility correlated with $\vec{X}$. 
    Let $\alpha > 0$, then for all but a $(1-1/\alpha)$ fraction of the $X_i$ the entropy loss is less than $\log{\alpha}/k$
\end{lemma}

\begin{proof} 

\begin{align}
    \acminent{\vec{X}}{Y} &= \Delta\\
    -\log{\Exlim{Y}{\max\limits_{\vec{x}} \Prob{\Vec{X} = \vec{x} \,|\, Y = y}}} &= \Delta\\
    \Exlim{Y}{\max\limits_{\vec{x}} \Prob{\Vec{X} = \vec{x} \,|\, Y = y}} &= 2^{-\Delta}\\
    \Problim{Y}{\max\limits_{\vec{x}} \Prob{\Vec{X} = \vec{x} \,|\, Y = y} \geq \alpha \cdot 2^{-\Delta}} &\leq \frac{1}{\alpha} \\
    \Problim{Y}{\log{\max\limits_{\vec{x}} \Prob{\Vec{X} = \vec{x} \,|\, Y = y}} \geq \log{\alpha} -\Delta} &\leq \frac{1}{\alpha} \\
    \Problim{Y}{-\log{\max\limits_{\vec{x}} \Prob{\Vec{X} = \vec{x} \,|\, Y = y}} < \Delta -\log{\alpha}} &\leq \frac{1}{\alpha} \\
    \Problim{y\leftarrow Y}{\minent{\vec{X}\, |\, Y=y} < \Delta -\log{\alpha}} &\leq \frac{1}{\alpha}
\end{align}
    
\end{proof}

\subsubsection{Upper bound for size of a Fuzzy Extractor}
In \cite{fuller2020fuzzy}, Fuller et al. show that the size of a fuzzy extractor can be upper bound in a general case. 
We restate their lemma here for the completeness of our main proof. 

\begin{lemma}[Lemma 5.2 in \cite{fuller2020fuzzy}]
    \label{lem:smallgeneralviable}
    Suppose $\mathcal{M}$ is $\zeroone{n}$ with the Hamming Metric, $\kappa \geq 2$, $0 \leq t \leq n/2$, and $\epsilon > 0$. 
    Suppose $(\mathsf{Gen, Rep})$ is a $(\mathcal{M,W},\kappa, t, \epsilon)$-fuzzy extractor for some distribution family $\mathcal{W}$ over $\mathcal{M}$. 
    Let $\tau = t/n$. 
    For any fixed $p$, there is a set $\mathsf{GoodKey}_p \subseteq \zeroone{\kappa}$ of size at least $2^{\kappa - 1}$ such that for every $\mathsf{key} \in \mathsf{GoodKey}_p$,
    \[
        \log{|\sbr{v \in \mathcal{M}|\prns{\mathsf{key}, p} \in \mathsf{supp}\prns{\mathsf{Gen}(v)}}|} \leq n \cdot h_2\prns{\frac{1}{2} - \tau} \leq n \cdot \prns{1 - \frac{2}{\ln{2}} \cdot \tau^{2}}, 
    \]   
    and, therefore, for any distribution $D_{\mathcal{M}}$ on $\mathcal{M}$, 
    \[
        \hart{D_{\mathcal{M}}|\mathsf{Gen}\prns{D_{\mathcal{M}}} = \prns{\mathsf{key}, p}} \leq n \cdot h_2\prns{\frac{1}{2} - \tau} \leq n \cdot \prns{1 - \frac{2}{\ln{2}} \cdot \tau^{2}}.
    \]   
\end{lemma}

\subsection{New Definitions}
Fuzzy extractors with quasipolynomial advice. 

\begin{definition}[Fuzzy Extractor with distributional advice]
Let $\mathcal{W}$ be a family of distributions indexed by $z$.  That is, one can denote each distribution in $\mathcal{W}$ as $W_Z$ which fully describes the probability mass function of $W$.  
For metric space $(\{0,1\}^n, \dis)$, a $(\{0,1\}^n, \mathcal{W}, \kappa, t, \epsilon, \ell)$-\emph{fuzzy extractor with distributional advice} is a triple of algorithms $(\gen, \rep, \aux)$ with the following properties:
\begin{enumerate} 
\itemsep0em
\item \textbf{Correctness} For all $w, w'$ such that $\dis(w, w')$, let $r, p \leftarrow \gen(w)$ then $\rep(w', p) = r$ with probability $1$. 
\item \textbf{Security} For all distributions $W_Z \in \mathcal{W}$, define $\advise \leftarrow \aux(Z)$, where $|\advise| \le \ell$, let $(R, P) \leftarrow \gen(W, \advise)$ and $U_\kappa$ be a uniformly distributed random variable over $\zo^\kappa$ it holds that $\Delta((R, P, Z), (U_\kappa, P, Z))\le \epsilon.$
\end{enumerate}
\end{definition}

\begin{definition}[Efficient Fuzzy Extractor in Known Distribution Setting]
Fix some distribution $W_Z$ Let $(\gen, \rep)$ be a $(\{0,1\}^n, \{W_Z\}, \kappa, t, \epsilon, \ell)$-fuzzy extractor.  The pair is \emph{efficient} if $\gen, \rep$ have the  additional property that they can be implemented by an algorithm that can be described in polynomial space (polynomial in the dimension $n$ of the metric space).
\end{definition}

\begin{lemma}
Let $\mathcal{W}$ be a distribution family indexed by the random family $Z$ and suppose that no $(\mathcal{M}, \mathcal{W}, \kappa, t, \epsilon, \ell)$-\emph{fuzzy extractor with distributional advice} exists for all $\ell = \poly(n)$.  Then there must be some distribution $W_Z \in \mathcal{W}$ for which no  $(\{0,1\}^n, \{W_Z\}, \kappa, t, \epsilon, \ell)$ efficient fuzzy extractor exists.
\label{lem:distributional advise suffices}
\end{lemma}
\begin{proof}[Proof of Lemma~\ref{lem:distributional advise suffices}]
We proceed by contrapositive.  Suppose that for every $W_Z\in\mathcal{W}$ there exists an efficient fuzzy extractor.  We denote these algorithms by $(\gen_Z, \rep_Z)$ respectively.  Across $Z$ let $\ell = \poly(n)$ represent the maximum space needed to implement some algorithm $\gen_Z$ or $\rep_Z$. We now describe how to build the fuzzy extractor $(\gen, \rep, \advise)$ with distributional advice.  Let $(\gen_Z, \rep_Z) \leftarrow \advise(Z)$ which has length at most $2\ell$.

Then define $\gen(x, C)$ as follows: 1) interpret $C$ as two circuits $\gen', \rep'$ and output $\gen'(x)$.  Define $\rep(x, p, C)$ interpret $C$ as two circuits $\gen', \rep'$ and output $\rep'(x', p)$.  It is clear that $(\gen, \rep, \advise)$ is a $(\mathcal{M}, \mathcal{W}, \kappa, t, \epsilon, 2\ell)$ fuzzy extractor with distributional advise.
\end{proof}


  % Proof for NegFE

\section{Proof of Main Theorem}

\subsection{Proof Sketch}
Our proof follows a fairly predictable structure. 
We begin by borrowing an existing result from \cite{fuller2020fuzzy} which gives an upperbound on the size of a set of viable points used by any good fuzzy extractor. You can find the theorem statement in Lemma \ref{lem:smallgeneralviable}.
We then further restrict this setting by showing that in order for an adversary to succeed on average they have to be able to align these viable points with a distribution that they have only a single sample and a polynomial length advice sting. 
We then argue that for large high entropy distributions this advice string can only reduce the entropy of a large fraction of viable points by a small amount. 
Then, we show that this small reduction of entropy for each point in the distribution means that on average the adversary cannot align the viable points with the distribution and there exists a distiguisher that can distinguish a uniform triple from a key triple. 

\subsection{Proof Setting}
Consider $\mathcal{W}$ such that each $W \in \mathcal{W}$ is a set of $2^{\phi}$ uniformly chosen independent random points in $\zeroone{n}$. 
Let $|\mathcal{W}| = r$ and let $Z \in \br{r}$ be an indexing variable for the selection of $W_Z$ from $\mathcal{W}$. 

In this proof the goal is to show that a fuzzy extractor cannot hope to hide the input point from an outside party. 
We will call the party that is building the Fuzzy Extractor the constructor and the party attempting to distiguish two settings the distinguisher. 
The Fuzzy Extractor is set up with an enrolled sample from our two stage sampling proceedure before the distiguishing game begins.
The game that the distinguisher plays is given one of two triples, real or random, do distinguish the realm to which the triple belongs. 
The real triple is the key value corresponding to the enrolled value in the fuzzy extractor, the public value produced by Gen, and the description of the distribution $W_Z$.
The random triple is the same except the key value corresponding to the enrolled value is substituted with a uniform value in the domain of the key values.

In this setting the constructor when creating the Fuzzy Extractor is aided by another party, we call this party the advisor. 
The advisor gets a full description of the distribution $W_Z$ and is allowed unbounded computation and time to produce an advice string $\mathsf{info}$. 
The advisor and constructor are unbounded in their computation and time, but the length of this string is required to be polynomial in the security parameter $\lambda$. 
The advice string is then communicated to the constructor, who also gets a sample from the distribution $w \in W_Z$. 
The constructor then is tasked with creating a Fuzzy Extractor, specifically choosing a public value $\mathsf{pub}$ which induces a partition over the space $\zeroone{n}$ and a key labeling of the partition which determines a key value for the initial sample.

\subsection{Maximum size of a Fuzzy Extractor}
From Lemma \ref{lem:smallgeneralviable}, we have the largest size of a set of viable points is $2^{\psi}$ where $\psi = n \cdot \prns{1 - \frac{2}{\ln{2}} \cdot \tau^{2}}$.

Now, it is clear from our setting that when building the Fuzzy Extractor the constructor has a single point $w$, a full description of the family of distributions $\mathcal{W}$, and the prepared advice string $\mathsf{info}$ about the specific selected distribution $W_Z$. 
We are primarily concerned with the ability of the constructor to align the points in the selected $W_Z$, with the partition induced by $\mathsf{pub}$. 
Clearly, if $\mathsf{info}$ is allowed to describe every point in $W_Z$ (or every point but $w$) then the constructor can include a point from $W_Z$ in each viable section of the partition induced by $\mathsf{pub}$. 
Fortuntely for us, the distribution has exponential entropy and the advice string has polynomial length so the advice string cannot describe the entire remainder of the distribution. 
The question remains, how much entropy remains in the distribution after seeing the advice string? 

Since each point in the distribution is independent and uniform in $\zeroone{n}$ the beginning  entropy (and min-entropy) of the distribution is $|W_Z| \cdot n$. 
Now we consider the advice string; since this string is allowed to arbitrarily depend on the distribution we can upperbound the min-entropy of the distribution conditioned on the advice string using a standard min-entropy argument found in Lemma \ref{lem:conditionalminentloss}. 
\[
    \acminent{W_Z}{\mathsf{info}} = |W_Z| \cdot 2^n - \log{|info|}
\]
  \bibliographystyle{alpha}
  \bibliography{negFE}
\end{document}